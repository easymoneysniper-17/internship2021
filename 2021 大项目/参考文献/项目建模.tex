\documentclass{article}
\usepackage{mathrsfs}
\usepackage{bm}
\usepackage{amsmath}
\usepackage{amsthm}
\usepackage{amssymb}
\usepackage{graphicx}
\usepackage{color}
\usepackage{comment}
%\include{macros}
%\usepackage{floatflt}
%\usepackage{graphics}
%\usepackage{epsfig}


\theoremstyle{definition}
\newtheorem{theorem}{Theorem}[section]
\newtheorem{lemma}[theorem]{Lemma}
\newtheorem{proposition}[theorem]{Proposition}
\newtheorem{corollary}[theorem]{Corollary}

\theoremstyle{definition}
\newtheorem*{defition}{Definition}
\newtheorem*{example}{Example}

\theoremstyle{remark}
\newtheorem*{remark}{Remark}
\newtheorem*{note}{Note}
\newtheorem*{exercise}{Exercise}

\setlength{\oddsidemargin}{-0.25 in}
\setlength{\evensidemargin}{-0.25 in} \setlength{\topmargin}{-0.25
in} \setlength{\textwidth}{7 in} \setlength{\textheight}{8.5 in}
\setlength{\headsep}{0.25 in} \setlength{\parindent}{0 in}
\setlength{\parskip}{0.1 in}

\newcommand{\homework}[5]{
\pagestyle{myheadings} \thispagestyle{plain}
\newpage
\setcounter{page}{1} \setcounter{section}{#5} \noindent
\begin{center}
\framebox{ \vbox{\vspace{2mm} \hbox to 6.28in { {\bf
Harvest~Fund~Project \hfill Version: 01} }
\vspace{6mm} \hbox to 6.28in { {\Large \hfill #1 \hfill} }
\vspace{6mm} \hbox to 6.28in { {\it Lecturer: #2 \hfill} }
\vspace{2mm} \hbox to 6.28in { {\it \hspace{13mm} #3 \hfill} }
\vspace{2mm} \hbox to 6.28in { {\it Student: #4 \hfill} }
\vspace{2mm} } }
\end{center}
\markboth{#1}{#1} \vspace*{4mm} }


\begin{document}

\homework{Alternative Energy Industry Chain Modelling}
{Junan Zhu \hspace{5mm} {\tt \quad jazhu@jsfund.cn}}
{}
{Zhenyu Jin \hspace{11mm} {\tt jzy20@mails.tsinghua.edu.cn}}{8}

\section*{Micro-Model Building}
\begin{itemize}
\item Notation:
\begin{table}[!ht]
	\centering
	\begin{tabular}{|c|c|c|}
		\hline
		Variable & Defination & Acquire Method\\
		\hline
		$c$ & The overall cost of mine from the upstream industry chain. & \\
		$K$ & The product capacity of fab planning. & \\
		$p(K)$ & The price and demand function(usually linear) &\\
		$D$ & Market demand, consists of two part $\overline{D}$ and $\tilde{D}$ &\\
		$\overline{D}$ & The stable demand in the market. & \\
		$\tilde{D}$ & The trendancy symbol of demand moving. &\\
		$h$ & Reputation loss since the lack of supplying. &\\
		$s$ & Reserving value of oversupply inventory. & \\
		\hline
	\end{tabular}
\end{table}
\item Consider the three-node supply chain for one industry, especially new-energy lottery industry.
\begin{equation*}
\begin{aligned}
	&\prod_{mid} = p\cdot min(D,K)-h(D-K)^{+} + s(K-D)^{+} -cK \\
	&\prod_{mid} = \int_0^K [p(D)\cdot D+s(K-D)]\cdot f(D)\,dD +\int_K^{\infty} [p(K)\cdot K-h(D-K)]\cdot f(D)dD-cK
\end{aligned}
\end{equation*}
\item Let the first order of the former equation to 0, thus can obtain that,
\begin{equation*}
\begin{aligned}
	\frac{\partial}{\partial K} \prod_{mid} &= [p(K)\cdot K -sK]\cdot f(K)+sF(K)+SK\cdot f(k)-p(K)K\cdot f(K) \\
		&+[p^{'}(K)\cdot K+p(K)]\cdot \overline{F(K)}+hK\cdot f(K)+h\overline{F(K)}-hK\cdot f(K)-c \\
		&=[p^{'}(K)\cdot K+p(K)]\cdot \overline{F(K)}+h\overline{F(K)}+sF(K)-c \\
		&=0
\end{aligned}
\end{equation*}
\item If the price function $p(D)$ maintain stable when the product has rigid market demand, the equation can be simplified to:
\begin{equation*}
\begin{aligned}
	& p\overline{F(K)}+h\overline{F(K)}+sF(K)=c \\
	& F_D(K)=\frac{p+h-c}{p+h-s} \\
	& K=F_D^{-1}(\frac{p+h-c}{p+h-s})
\end{aligned}
\end{equation*}
\end{itemize}































\end{document}